\begin{abstract}

HPC developers are abandoning POSIX because the synchronization and
serialization overheads of providing strong consistency and durability are too
costly -- and often unnecessary -- for their applications.  Unfortunately,
designing near-POSIX file systems excludes applications that rely on strong
consistency or durability, forcing developers to re-write their applications or
deploy them on a different system.  We present a file system and API that lets
clients specify their consistency/durability requirements and assign them to
subtrees in the namespace, allowing users to optimize subtrees within the same
namespace for different workloads.  We draw conclusions about the performance
impact of unexplored consistency/durability metadata designs and show that
maintaining strong consistency can cause about a 100\(\times\) slow down
compared to relaxed consistency and no durability. Comparatively, merging
updates after a period of relaxed consistency (less than a 10\(\times\) slow
down) and maintaining durability (about a 10\(\times\) slow down) have more
reasonable costs.

\end{abstract}
