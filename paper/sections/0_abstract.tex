\begin{abstract}

HPC and data center scale application developers are abandoning POSIX IO
because file system metadata synchronization and serialization overheads of
providing strong consistency and durability are too costly -- and often
unnecessary -- for their applications.  Unfortunately, designing file systems
with weaker consistency or durability semantics excludes applications that rely on
stronger guarantees, forcing developers to re-write their applications or
deploy them on a different system.  We present a framework and API that lets
\oldcomment{clients}\newcomment{administrators} specify their
consistency/durability requirements and dynamically assign them to subtrees in
the same namespace, allowing administrators to optimize subtrees over time and
space for different workloads. \oldcomment{By custom fitting a subtree to a
create-heavy application,} We show similar speedups to related work but more
importantly, \oldcomment{our prototype can custom fit subtrees in the same
namespace to applications common in large data centers, }\newcomment{ we show
performance improvements when we custom fit subtree semantics to applications
such as checkpoint-restart (91.7\(\times\) speedup), user home directories
(0.03 standard deviation from optimal), and users checking for partial results
(2\% \oldcomment{clients}\newcomment{users} overhead).} \oldcomment{ and can
scale to 2\(\times\) as many clients when compared to our baseline system.}

\end{abstract}

%Users can mount multiple systems in the
%global namespace but this means (1) provisioning separate storage clusters and
%(2) manually moving data across system boundaries.  

%We confirm the performance benefits of
%techniques presented in related work but also explore new
%consistency/durability metadata designs, all integrated over the same storage
%system.  
