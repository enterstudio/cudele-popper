\section{Related Work} 
\label{sec:related-work}

% General
The bottlenecks associated with accessing POSIX file system metadata are not limited
to HPC workloads and the same challenges that plagued these systems for years are
finding their way into the cloud. Workloads that deal with many small files
({\it e.g.}, log processing and database
queries~\cite{thusoo:sigmod2010-facebook-infrastructure}) and large numbers of
simultaneous clients ({\it e.g.}, MapReduce
jobs~\cite{mckusick:acm2010-gfs-evolution}), are subject to the scalability of
the metadata service. The biggest challenge is that whenever a file
is touched the client must access the file's metadata and maintaining a file
system namespace imposes small, frequent accesses on the underlying storage
system~\cite{roselli:atec2000-FS-workloads}.  Unfortunately, scaling file
system metadata is a well-known problem and solutions for scaling data IO do
not work for metadata IO~\cite{roselli:atec2000-FS-workloads,
abad:techreport2012-fstrace, abad:ucc2012-mimesis,
alam:pdsw2011-metadata-scaling, weil:osdi2006-ceph}. There are two approaches
for improving the performance of metadata access.

\subsection{Metadata Load Balancing}

% approaches to load balancing (strong = more resuorces per unit work, weak =
% fixed resource per work unit)
% CITEME
One approach for improving metadata performance and scalability is to alleviate
overloaded servers by load balancing metadata IO across a cluster. Common
techniques include partitioning metadata when there are many writes and
replicating metadata when there are many reads. For example, IndexFS~\cite{ren:sc2014-indexfs} partitions
directories and clients write to different partitions by grabbing leases and
caching ancestor metadata for path traversal; it does well for strong scaling
because servers can keep more inodes in the cache which results in less RPCs.
Alternatively, ShardFS replicates directory state so servers do not need to
contact peers for path traversal; it does well for read workloads because all
file operations only require 1 RPC and for weak scaling because requests will
never incur extra RPCs due to a full cache.  CephFS employs both techniques to
a lesser extent; directories can be replicated or sharded but the caching and
replication policies do not change depending on the balancing technique.
Despite the performance benefits these techniques add complexity and jeopardize
the robustness and performance characteristics of the metadata service because
the systems now need (1) policies to guide the migration decisions and (2)
mechanisms to address inconsistent states across servers.

% policies
Setting policies for migrations is arguably more difficult than adding the
migration mechanisms themselves.  For example, IndexFS and CephFS use the
GIGA+~\cite{patil:fast2011-giga} technique for partitioning directories at a predefined
threshold and using lazy synchronization to redirect queries to the server that
``owns" the targeted metadata.  Determining when to partition directories and
when to migrate the directory fragments are policies that vary between systems:
GIGA+ partitions directories when the size reaches a certain number of files
and migrates directory fragments immediately; CephFS partitions directories
when they reach a threshold size or when the write temperature reaches a
certain value and migrates directory fragments when the hosting server has more
load than the other servers in the metadata cluster. Another policy is when and
how to replicate directory state; ShardFS replicates immediately and
pessimistically while CephFS replicates only when the read temperature reaches
a threshold.  There is a wide range of policies and it is difficult to maneuver
tunables and hard-coded design decisions.

% addressing inconsistency
In addition to the policies, distributing metadata across a cluster requires
distributed transactions and cache coherence protocols to ensure strong
consistency ({e.g.}, POSIX).  For example, ShardFS pessimistically replicates
directory state and uses optimistic concurrency control for conflicts; namely
it does the operation and if there is a conflict at verification time it falls
back to two-phase locking.  Another example is IndexFS's inode cache which
reduces RPCs by caching ancestor paths -- the locality of this cache can be
thrashed by random reads but performs well for metadata writes. For
consistency, writes to directories in IndexFS block until the lease expires
while writes to directories in ShardFS are slow for everyone as it either
requires serialization or locking with many servers; reads in IndexFS are
subject to cache locality while reads in ShardFS always resolve to 1 RPC.
Another example of the overheads of addressing inconsistency is how CephFS
maintains client sessions and inode caches for capabilities (which in turn make
metadata access faster). When metadata is exchanged between metadata servers
these sessions/caches must be flushed and new statistics exchanged with a
scatter-gather process; this halts updates on the directories and blocks until
the authoritative metadata server responds.  These protocols are discussed in
more detail in the next section but their inclusion here is a testament to the
complexity of migrating metadata.

%For metadata writes it does a distributed transaction; monotonic writes with
%concurrent clients fail and do pessimistic locking through a mediated lock
%server to ensure strong consistency, non-monotonic writes grab locks at every
%server. Zooming in on monotonic writes: if permissions increase it executes on
%a primary then non-primary, if permissions decrease it executes on all
%non-primary then on primary.

The conclusion we have drawn from this related work is that metadata protocols
have a bigger impact on performance and scalability than load balancing.  
Understanding these protocols helps load balancing and gives us a better
understanding of the metrics we should use to make migration decisions ({\it
e.g.}, which operations reflect the state of the system), what types of
requests cause the most load, and how an overloaded system reacts ({\it e.g.},
increasing latencies, lower throughput, etc.).

\subsection{Relaxing POSIX}

% POSIX 
POSIX workloads require strong consistency and many file systems improve
performance by reducing the number of remote calls per operation ({\it i.e.}
RPC amplification). As discussed in the previous section, caching with leases and
replication are popular approaches to reducing the overheads of path traversals
but their performance is subject to cache locality and the amount of available
resources, respectively; for random workloads larger than the cache extra RPCs
hurt performance~\cite{ren:sc2014-indexfs, weil:sc2004-dyn-metadata} and for write heavy workloads with more
resources the RPCs for invalidations are harmful. Another approach to reducing
RPCs is to use leases or capabilities.  


%IndexFS aggressively caches paths and
%handles permissions by handing out leases for metadata writes; metadata may
%only be modified when all leases have expired. 
%IndexFS~\cite{ren:sc2014-indexfs} aggressively caches pathnames and their
%permissions on the client servers. Modifications to metadata cached by clients
%is delayed until all client leases have expired. This reduces the RPC
%amplification to 1 when mutatating directory metadata ({\it e.g.,}
%\texttt{mkdir}, \texttt{chmod}, etc.) because clients are not querying metadata servers for
%path traversals. The disadvantage of this approach is the high latency of
%mutation operations (reads to filenames).  ShardFS~\cite{xiao:socc2015-shardfs}
%replicates metadata (specifically directory lookup state) across the metadata server
%cluster, reducing the RPC amplification to 1 for file operations ({e.g.,}
%\texttt{stat}, \texttt{chmod}, \texttt{chown}, etc.). Modifications to the
%directory lookup state are done with optimistic concurrecny control and fall
%back to retry if verification fails. The disadvantage of this appraoch is the
%high number of RPCs for maintaining directory metadata mutations (writes to
%directories).  CephFS also maintains an inode cache but its notion of leases
%are much shorter, on the order of microseconds.

% Non POSIX
High performance computing has unique requirements for file systems ({\it
e.g.}, fast creates) and well-defined workloads ({\it e.g.}, workflows) that
make relaxing POSIX sensible.  BatchFS assumes the application coordinates
accesses to the namespace, so the clients can batch local operations and merge
with a global namespace image lazily. Similarly, DeltaFS eliminates RPC
traffic using subtree snapshots for non-conflicting workloads and middleware
for conflicting workloads. MarFS gives users the ability to lock
``project directories" and allocate GPFS clusters for demanding metadata
workloads. TwoTiers eliminates high-latencies by storing metadata in a flash
tier; applications lock the namespace so that metadata can be accessed more quickly.
Unfortunately, decoupling the namespace has costs: (1) merging metadata state
back into the global namespace is slow; (2) failures are local to the failing
node; and (3) the systems are not backwards compatible. 

% NON POSIX
%Decoupling the namespaces has many advantages, including improved scalability,
%higher resource utilization, and better performance.  BatchFS and DeltaFS are
%near-POSIX filesystems that give clients the ability to decouple subtrees from
%the namespace so that the applications can execute metadata operations without
%synchronization and serialization.  These operations are applied to a local
%snapshot of the file system namespace and conflicts are resolved either by the
%application or by an external service . Applications link into a metadata
%server library to reduce resource utilization and code paths (e.g., no daemons
%and less interprocess communication). 

For (1), state-of-the-art systems manage consistency in non-traditional ways:
IndexFS maintains the global namespace but blocks operations from other clients
until the first client drops the lease, BatchFS does operations on a snapshot
of the namespace and merges batches of operations into the global namespace,
and DeltaFS never merges back into the global namespace. The merging for
BatchFS is done by an auxiliary metadata server running on the client and
conflicts are resolved by the application. Although DeltaFS never explicitly
merges, applications needing some degree of ground truth can either manage
consistency themselves on a read or add a bolt-on service to manage the
consistency.

For (2), if the client fails and stays down, all metadata operations on the
decoupled namespace are lost. If the client recovers, the on-disk structures
(for BatchFS and DeltaFS this is the SSTables used in TableFS) can be
recovered. In other words, the clients have state that cannot be recovered if
the node stays failed and any progress will be lost. This scenario is a
disaster for checkpoint-restart where missed cycles may cause the checkpoint to
bleed over into computation time.

For (3), decoupled namespace approaches sacrifice POSIX going as far as
requiring the application to link against the systems they want to talk to. In
today's world of software defined caching, this can be a problem for large data
centers with many types and tiers of storage. Despite well-known performance
problems POSIX and REST are the dominant APIs for data transfer.
