% Template: http://www.acm.org/publications/proceedings-template
\documentclass[onecolumn,conference]{IEEEtran}
\usepackage{booktabs} % For formal tables
\usepackage{minted}
\usepackage{graphicx}
\usepackage{arydshln}
\usepackage{subcaption}
\usepackage{times}
\usepackage{hyperref}
\hypersetup{
    colorlinks=true,
    linkcolor=blue,
    filecolor=magenta,      
    urlcolor=cyan,
}

\begin{document}
\title{Cover Letter}
\maketitle

We thank the reviewers for their hard work and genuinely helpful suggestions.
In addition to this cover letter, we have posted a PDF online explicitly
showing additions in \textcolor{red}{\textbf{red}} and deletions in
\textcolor{blue}{\textbf{blue}}. Three issues were common across reviewer
feedback; Reviewer 1 and 3's comments are addressed in issues I and II,
Reviewer 2's comments are addressed in issues I and III, and Reviewer 4's
comments are addressed in issues II and III. 

%Use Cases R2, R4
%- Spark discussion R2
%- more emphasis on parallel and distributed computing themes R4
\section*{Issue I: Synthetic and/or Irrelevant Use Cases (Reviewers 2 and 4)}

To show the relevance of the use cases, we re-wrote Section~{\S}V-B to
highlight examples from HPC and cloud workloads that would benefit from using
Cudele. For HPC we focus on user home directories used for experiments and for
the cloud we focus on the Hadoop and Spark runtimes (as suggested by Reviewer
2). To connect the problematic use cases to Cudele, we added \textbf{Cudele
setup} headers in Section~{\S}V-B to show which subtree semantics accomodate
these workloads in the global namespace.  By showing these setups, we hope to
demonstrate that different applications can use a file system in parallel with
different consistency/durability semantics.  Although we do not actually run
the workloads on our prototype, hopefully our changes show how the use cases,
and the synthetic benchmarks we used to represent them, are indeed relevant to
parallel and distributed computing.

To align the paper to the themes of the conference, we changed Section~{\S}V-A
and Figures 6a, 6b, and 6c to emphasize that decoupled namespaces facilitate
client-driven parallelism. Clients detach subtrees form the global namespaces
and do operations in-parallel to their local disk. We also hope that changes to
the use cases above motivates the need for robust distributed storage systems
that can handle today's applications.  So while the paper is storage-centric,
we try to show that the workloads it supports are highly distributed and need a
flexible solution like Cudele.  



%Mixing contributions and future work (R1)
%Statements in Introduction do not align with Evaluation (R1, R2, R3)
%- quantify performance speedups R3
%Evaluation Structure (R2)
%- remove major takeaways and cross refercnes
%Will not address:
%- cost of dynamically changing consistency/durability not presented R1
\section*{Issue II: Structure and Layout of Evaluation (All Reviewers)}

All reviewers note that the results and contributions we cite in the
introduction are not validated, explained, or even mentioned in the evaluation.
One confusing component of this issue is that we mixed future work with the
contributions of the paper; to make the contributions more explicit we remove
future work from the introduction and add it to Section~{\S}VI; we do not
attempt to validate or prove any of the statements about the benefits of
changing consistency and durability properties of a subtree dynamically, such
as saving resource provisioning costs.  To further clarify the evaluation, we:

\begin{itemize}

  \item explicitly state the Cudele setups and subtree configurations in bold
in Section~{\S}V-A. Hopefully this helps the reader keep track of use cases and
demonstrates how Cudele might be able to help in these scenarios. 

and include baseline values in the
text so readers can quickly calculate raw numbers from our speedup/slowdown
graphs. We also make sure to connect all speedups from the introduction to the
evaluation. We hope this clarifies confusion about what we are comparing to
when we reference speedups and slowdowns.

  \item re-organized the sections and removed cross-references. Experiments are
now self-contained so the reader can see the effects of different API
configurations and we do a better job of explaining how results are derived. 

  \item removed the ``major takeaways". We deleted the headings but moved the
text to the beginning of each section to motivate why we are doing each
experiment.  We also add insights into the results by analyzing the raw numbers
we observe in comparison to hardware capabilities.

  \item re-labeled all graphs with the same metric (throughput
slowdown/speedup) and aligned Section~{\S}V with the graphs in Section~{\S}II.
We hope this eliminates more of the confusion triggered by the
cross-referencing we did, referenced by Reviewers 2 and 3.

\end{itemize}

%Experimental Setup (R1, R2, R3)
%- code R1
%- servers, network, storage R1, R3
\section*{Issue III: No Experimental Setup (Reviewers 1, 2, and 3)}

We apologize for the omission of experimental setup and source code details. To
provide a more comprehensive view of our experiments, we added text to the
paper and made paper artifacts available online. For the experimental setup, we
describe the cluster setup (hardware, software, etc.) in Section~{\S}V.  We
have also made the infrastructure code available and added links after each
figure to show exactly how experiments are run.  This infrastructure code
contains scripts to deploy the system, run experiments, and gather results.
This process follows the Popper
Convention\footnote{http://falsifiable.us/}~\cite{jimenez_popper_2016}, which
aims to make research reproducible.  We also made the source code available
online and a link is provided in a footnote in Section~{\S}V. Finally, we add
details about which tools and classes we modified in Section~{\S}IV to address
Reviewer 1's questions about the framework implementation.



\section*{Cosmetic Fixes}
%Terminology
%- users vs. clients R2
\begin{itemize}
  \item Reviewer 1: fixed user vs. client terminology in Section~{\S}III
  \item Reviewer 1: clarified that the cost of dynamically changing semantics is future work Sections~{\S}III and ~{\S}V.E
  \item Reviewer 1: added source code pointer in Section~{\S}IV
  \item Reviewer 2: remove major takeways in Section~{\S}V
  \item Reviewer 2: remove major cross-references in Section~{\S}V
  \item Reviewer 3: quantified speedups with figure annotations in Section~{\S}V
  \item Reviewers 1 and 3: add servers, network, and storage setups in Section~{\S}V
  \item Reviewer 4: add new section ({\S}V.F) describing how Cudele would work with a system like Spark
\end{itemize}

\bibliographystyle{IEEEtran}
\bibliography{IEEEabrv,paper}
\end{document}
